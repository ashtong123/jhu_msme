\documentclass[12pt, letterpaper]{article}
\begin{document}
\setlength\parindent{0pt}
\textbf{
1.1.13\\
Given:
\[ y' = y - y^2,\; y = \frac{1}{1+ce^{-x}},\; y(0) = 0.25 \]
(a) Verify that $y$ is a solution of the ODE.\\
(b) Determine from $y$ the particular solution of the IVP.\\
(c) Graph the solution of the IVP.\\
}

\vspace{5mm}
\textit{Solution to (a):}\\
Start by differentiating $y$, which can be done using the chain rule:
\[ \frac{d}{dx}[f(g(x))] = f'(g(x))*g'(x) \]
In this case, $g$ is the function of x:
\[ g(x) = 1+ce^{-x} \]
And $f$ is therefore the function of g:
\[ f(g(x)) = \frac{1}{g(x)} = (g(x))^{-1} \]
First, we can calculate the derivative of f with respect to g,
which can be solved using the power rule:
\[ \frac{dx^n}{dx} = nx^{n-1} \]
Thus:
\[ \frac{df(g(x))}{dg(x)} = -1(g(x))^{-2} \]
And, the derivative of g with respect to x, which can be solved using the chain
rule as well:
\[ \frac{d(g(x))}{dx} = -ce^{-x} \]
Thus, the derivative $y'$ is:
\[ y' = -1(1+ce^{-x})^{-2}*(-ce^{-x}) \] 
Plugging $y$ and $y'$ into the original ODE:
\[ -1(1+ce^{-x})^{-2}*(-ce^{-x}) = (1+ce^{-x})^{-1} - (1+ce^{-x})^{-2} \] 
We can factor and cancel out $(1+ce^{-x})^{-2}$ from both sides of the equation:
\[ (1+ce^{-x})^{-2}*(ce^{-x}) = (1+ce^{-x})^{-2}*((1+ce^{-x}) - 1) \]
Leaving us with:
\[ ce^{-x} = 1 + ce^{-x} - 1 \]
\[ ce^{-x} = ce^{-x} \]
The last statement confirming that $y$ is a solution to the ODE.

\vspace{5mm}
\textit{Solution to (b):}\\
Given $y = \frac{1}{1+ce^{-x}}$:
\[ 0.25 = \frac{1}{1+ce^{-0}} \]
\[ 1+ce^{-0} = \frac{1}{0.25} = 1 + c = 4 \]
\[ c = 3 \]
Therefore, the general solution is:
\[ y = \frac{1}{1+3e^{-x}} \]

\vspace{5mm}
\textit{Solution to (c):}\\


\end{document}