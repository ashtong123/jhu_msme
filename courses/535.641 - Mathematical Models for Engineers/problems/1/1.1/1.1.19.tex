\documentclass[12pt, letterpaper]{article}
\begin{document}
\setlength\parindent{0pt}
\textbf{
1.1.19\\
Free fall. 
}
In dropping a stone or an iron ball, air
resistance is practically negligible. Experiments
show that the acceleration of the motion is constant
(equal to $g = 9.80 m/sec^2$ called the
acceleration of gravity). Model this as an ODE for
$y(t)$, the distance fallen as a function of time $t$. If the
motion starts at time $t = 0$ from rest (i.e., with velocity
$v = y' = 0$), show that you obtain the familiar law of
free fall
\[ y = \frac{1}{2}gt^2 \]
We know that the acceleration is a constant, and that the acceleration
is the second derivative of the position with respect to time, thus:
\[ y'' = 9.80 \]
Integrating the above, we can determine the velocity as a function of time:
\[ y' = 9.80t + c \]
Given that $y'(0) = 0$, then we know that $c = 0$, so:
\[ y' = 9.80t \]
We can integrate again to determine the position as a function of time:
\[ y = \frac{1}{2}9.80t^2 + c \]
Where $c$ is the initial position of the falling object, if we assume $y(0) = 0$
\[ \]
\end{document}