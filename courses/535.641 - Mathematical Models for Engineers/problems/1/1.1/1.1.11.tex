\documentclass[12pt, letterpaper]{article}
\begin{document}
\setlength\parindent{0pt}
\textbf{
1.1.11\\
Given:
\[ y' = y + e^x,\; y = (x + c)e^x,\; y(0) = \frac{1}{2} \]
(a) Verify that $y$ is a solution of the ODE.\\
(b) Determine from $y$ the particular solution of the IVP.\\
(c) Graph the solution of the IVP.\\
}

\vspace{5mm}
\textit{Solution to (a):}\\
We can begin by differentiating $y$ to determine $y'$:
\[ \frac{d}{dx}[(x + c)e^x] \]
This derivative can be calculated using the product rule:
\[ \frac{d}{dx}uv = u'v + uv' \]
Where $u = x + c$ and $v = e^x$. Thus we determine $u'$ and $v'$:
\[ \frac{d}{dx}[x + c] = 1 \]
\[ \frac{d}{dx}[e^x] = e^x \]
Then, we can solve for $y'$:
\[ y' = 1*e^x + (x+c)*e^x \]
Next, we can substitute our values of $y$ and $y'$ back into the original ODE
and see if $y$ is a valid solution:
\[ e^x + (x + c)e^x = (x + c)e^x + e^x \]
It is obvious that the equation above is true, therefore $y$ is a valid solution
to the ODE

\vspace{5mm}
\textit{Solution to (b):}\\
We can substitute $x_0$ and $y_0$ into y and solve for $c$ to get the particular
solution:
\[ \frac{1}{2} = (0 + c)e^0 \] 
\[ \frac{1}{2} = c \]
Therefore the particular solution is:
\[ y = (x + \frac{1}{2})e^x \]

\vspace{5mm}
\textit{Solution to (c):}\\
The graph of the particular solution is in this folder as 1.1.11_graph.png

\end{document}